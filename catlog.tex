\documentclass{amsart}
\usepackage{mathpartir,cancel}
\newif\ifcref\creftrue
\usepackage{amssymb,amsmath,stmaryrd,mathrsfs}

%% Set this to true before loading if we're using the TAC style file.
%% Note that eventually, TAC requires everything to be in one source file.
\def\definetac{\newif\iftac}    % Can't define a \newif inside another \if!
\ifx\tactrue\undefined
  \definetac
  %% Guess whether we're using TAC by whether \state is defined.
  \ifx\state\undefined\tacfalse\else\tactrue\fi
\fi

% Similarly detect beamer
\def\definebeamer{\newif\ifbeamer}
\ifx\beamertrue\undefined
  \definebeamer
  %% Guess whether we're using Beamer by whether \uncover is defined.
  \ifx\uncover\undefined\beamerfalse\else\beamertrue\fi
\fi

% And cleveref
\def\definecref{\newif\ifcref}
\ifx\creftrue\undefined
  \definecref
  % Default to false
  \creffalse
\fi

\iftac\else\usepackage{amsthm}\fi
\usepackage[all,2cell]{xy}
%\UseAllTwocells
\usepackage{tikz}
\usetikzlibrary{arrows}
\tikzset{ed/.style={auto,inner sep=0pt,font=\scriptsize}} %edges
\ifbeamer\else
  \usepackage{enumitem}
  \usepackage{xcolor}
  \definecolor{darkgreen}{rgb}{0,0.45,0} 
  \iftac\else\usepackage[pagebackref,colorlinks,citecolor=darkgreen,linkcolor=darkgreen]{hyperref}\fi
\fi
\usepackage{mathtools}          % for all sorts of things
\usepackage{graphics}           % for \scalebox, used in \widecheck
\usepackage{ifmtarg}            % used in \jd
\usepackage{microtype}
%\usepackage{color,epsfig}
%\usepackage{fullpage}
%\usepackage{eucal}
%\usepackage{wasysym}
%\usepackage{txfonts}            % for \invamp, or for the nice fonts
\usepackage{braket}             % for \Set, etc.
\let\setof\Set
\usepackage{url}                % for citations to web sites
\usepackage{xspace}             % put spaces after a \command in text
%\usepackage{cite}               % compress and sort grouped citations (only use with numbered citations)
\ifcref\usepackage{cleveref,aliascnt}\fi

%% If you want to use biblatex, e.g. if a journal requires (Author name YEAR) citations.
% \usepackage[style=authoryear,
%  backref=true,
%  maxnames=4,
%  maxbibnames=99,
%  uniquename=false,
%  firstinits=true
% ]{biblatex}
% \addbibresource{all.bib}

% \let\cite\parencite
% \DeclareNameAlias{sortname}{last-first}

\makeatletter
\let\ea\expandafter

%% Defining commands that are always in math mode.
\def\mdef#1#2{\ea\ea\ea\gdef\ea\ea\noexpand#1\ea{\ea\ensuremath\ea{#2}\xspace}}
\def\alwaysmath#1{\ea\ea\ea\global\ea\ea\ea\let\ea\ea\csname your@#1\endcsname\csname #1\endcsname
  \ea\def\csname #1\endcsname{\ensuremath{\csname your@#1\endcsname}\xspace}}

%% WIDECHECK
\DeclareRobustCommand\widecheck[1]{{\mathpalette\@widecheck{#1}}}
\def\@widecheck#1#2{%
    \setbox\z@\hbox{\m@th$#1#2$}%
    \setbox\tw@\hbox{\m@th$#1%
       \widehat{%
          \vrule\@width\z@\@height\ht\z@
          \vrule\@height\z@\@width\wd\z@}$}%
    \dp\tw@-\ht\z@
    \@tempdima\ht\z@ \advance\@tempdima2\ht\tw@ \divide\@tempdima\thr@@
    \setbox\tw@\hbox{%
       \raise\@tempdima\hbox{\scalebox{1}[-1]{\lower\@tempdima\box
\tw@}}}%
    {\ooalign{\box\tw@ \cr \box\z@}}}

%% SIMPLE COMMANDS FOR FONTS AND DECORATIONS

\newcount\foreachcount

\def\foreachletter#1#2#3{\foreachcount=#1
  \ea\loop\ea\ea\ea#3\@alph\foreachcount
  \advance\foreachcount by 1
  \ifnum\foreachcount<#2\repeat}

\def\foreachLetter#1#2#3{\foreachcount=#1
  \ea\loop\ea\ea\ea#3\@Alph\foreachcount
  \advance\foreachcount by 1
  \ifnum\foreachcount<#2\repeat}

% Script: \sA is \mathscr{A}
\def\definescr#1{\ea\gdef\csname s#1\endcsname{\ensuremath{\mathscr{#1}}\xspace}}
\foreachLetter{1}{27}{\definescr}
% Calligraphic: \cA is \mathcal{A}
\def\definecal#1{\ea\gdef\csname c#1\endcsname{\ensuremath{\mathcal{#1}}\xspace}}
\foreachLetter{1}{27}{\definecal}
% Bold: \bA is \mathbf{A}
\def\definebold#1{\ea\gdef\csname b#1\endcsname{\ensuremath{\mathbf{#1}}\xspace}}
\foreachLetter{1}{27}{\definebold}
% Blackboard Bold: \dA is \mathbb{A}
\def\definebb#1{\ea\gdef\csname d#1\endcsname{\ensuremath{\mathbb{#1}}\xspace}}
\foreachLetter{1}{27}{\definebb}
% Fraktur: \fa is \mathfrak{a}, except for \fi; \fA is \mathfrak{A}
\def\definefrak#1{\ea\gdef\csname f#1\endcsname{\ensuremath{\mathfrak{#1}}\xspace}}
\foreachletter{1}{9}{\definefrak}
\foreachletter{10}{27}{\definefrak}
\foreachLetter{1}{27}{\definefrak}
% Sans serif: \ia is \mathsf{a}, except for \if and \in
\def\definesf#1{\ea\gdef\csname i#1\endcsname{\ensuremath{\mathsf{#1}}\xspace}}
\foreachletter{1}{6}{\definesf}
\foreachletter{7}{14}{\definesf}
\foreachletter{15}{27}{\definesf}
\foreachLetter{1}{27}{\definesf}
% Bar: \Abar is \overline{A}, \abar is \overline{a}
\def\definebar#1{\ea\gdef\csname #1bar\endcsname{\ensuremath{\overline{#1}}\xspace}}
\foreachLetter{1}{27}{\definebar}
\foreachletter{1}{8}{\definebar} % \hbar is something else!
\foreachletter{9}{15}{\definebar} % \obar is something else!
\foreachletter{16}{27}{\definebar}
% Tilde: \Atil is \widetilde{A}, \atil is \widetilde{a}
\def\definetil#1{\ea\gdef\csname #1til\endcsname{\ensuremath{\widetilde{#1}}\xspace}}
\foreachLetter{1}{27}{\definetil}
\foreachletter{1}{27}{\definetil}
% Hats: \Ahat is \widehat{A}, \ahat is \widehat{a}
\def\definehat#1{\ea\gdef\csname #1hat\endcsname{\ensuremath{\widehat{#1}}\xspace}}
\foreachLetter{1}{27}{\definehat}
\foreachletter{1}{27}{\definehat}
% Checks: \Achk is \widecheck{A}, \achk is \widecheck{a}
\def\definechk#1{\ea\gdef\csname #1chk\endcsname{\ensuremath{\widecheck{#1}}\xspace}}
\foreachLetter{1}{27}{\definechk}
\foreachletter{1}{27}{\definechk}
% Underline: \uA is \underline{A}, \ua is \underline{a}
\def\defineul#1{\ea\gdef\csname u#1\endcsname{\ensuremath{\underline{#1}}\xspace}}
\foreachLetter{1}{27}{\defineul}
\foreachletter{1}{27}{\defineul}

% Particular commands for typefaces, sometimes with the first letter
% different.
\def\autofmt@n#1\autofmt@end{\mathrm{#1}}
\def\autofmt@b#1\autofmt@end{\mathbf{#1}}
\def\autofmt@d#1#2\autofmt@end{\mathbb{#1}\mathsf{#2}}
\def\autofmt@c#1#2\autofmt@end{\mathcal{#1}\mathit{#2}}
\def\autofmt@s#1#2\autofmt@end{\mathscr{#1}\mathit{#2}}
\def\autofmt@f#1\autofmt@end{\mathsf{#1}}
\def\autofmt@k#1\autofmt@end{\mathfrak{#1}}
% Particular commands for decorations.
\def\autofmt@u#1\autofmt@end{\underline{\smash{\mathsf{#1}}}}
\def\autofmt@U#1\autofmt@end{\underline{\underline{\smash{\mathsf{#1}}}}}
\def\autofmt@h#1\autofmt@end{\widehat{#1}}
\def\autofmt@r#1\autofmt@end{\overline{#1}}
\def\autofmt@t#1\autofmt@end{\widetilde{#1}}
\def\autofmt@k#1\autofmt@end{\check{#1}}

% Defining multi-letter commands.  Use this like so:
% \autodefs{\bSet\cCat\cCAT\kBicat\lProf}
\def\auto@drop#1{}
\def\autodef#1{\ea\ea\ea\@autodef\ea\ea\ea#1\ea\auto@drop\string#1\autodef@end}
\def\@autodef#1#2#3\autodef@end{%
  \ea\def\ea#1\ea{\ea\ensuremath\ea{\csname autofmt@#2\endcsname#3\autofmt@end}\xspace}}
\def\autodefs@end{blarg!}
\def\autodefs#1{\@autodefs#1\autodefs@end}
\def\@autodefs#1{\ifx#1\autodefs@end%
  \def\autodefs@next{}%
  \else%
  \def\autodefs@next{\autodef#1\@autodefs}%
  \fi\autodefs@next}

%% FONTS AND DECORATION FOR GREEK LETTERS

%% the package `mathbbol' gives us blackboard bold greek and numbers,
%% but it does it by redefining \mathbb to use a different font, so that
%% all the other \mathbb letters look different too.  Here we import the
%% font with bb greek and numbers, but assign it a different name,
%% \mathbbb, so as not to replace the usual one.
\DeclareSymbolFont{bbold}{U}{bbold}{m}{n}
\DeclareSymbolFontAlphabet{\mathbbb}{bbold}
\newcommand{\dDelta}{\ensuremath{\mathbbb{\Delta}}\xspace}
\newcommand{\done}{\ensuremath{\mathbbb{1}}\xspace}
\newcommand{\dtwo}{\ensuremath{\mathbbb{2}}\xspace}
\newcommand{\dthree}{\ensuremath{\mathbbb{3}}\xspace}

% greek with bars
\newcommand{\albar}{\ensuremath{\overline{\alpha}}\xspace}
\newcommand{\bebar}{\ensuremath{\overline{\beta}}\xspace}
\newcommand{\gmbar}{\ensuremath{\overline{\gamma}}\xspace}
\newcommand{\debar}{\ensuremath{\overline{\delta}}\xspace}
\newcommand{\phibar}{\ensuremath{\overline{\varphi}}\xspace}
\newcommand{\psibar}{\ensuremath{\overline{\psi}}\xspace}
\newcommand{\xibar}{\ensuremath{\overline{\xi}}\xspace}
\newcommand{\ombar}{\ensuremath{\overline{\omega}}\xspace}

% greek with tildes
\newcommand{\altil}{\ensuremath{\widetilde{\alpha}}\xspace}
\newcommand{\betil}{\ensuremath{\widetilde{\beta}}\xspace}
\newcommand{\gmtil}{\ensuremath{\widetilde{\gamma}}\xspace}
\newcommand{\phitil}{\ensuremath{\widetilde{\varphi}}\xspace}
\newcommand{\psitil}{\ensuremath{\widetilde{\psi}}\xspace}
\newcommand{\xitil}{\ensuremath{\widetilde{\xi}}\xspace}
\newcommand{\omtil}{\ensuremath{\widetilde{\omega}}\xspace}

% MISCELLANEOUS SYMBOLS
\let\del\partial
\mdef\delbar{\overline{\partial}}
\let\sm\wedge
\newcommand{\dd}[1]{\ensuremath{\frac{\partial}{\partial {#1}}}}
\newcommand{\inv}{^{-1}}
\newcommand{\dual}{^{\vee}}
\mdef\hf{\textstyle\frac12 }
\mdef\thrd{\textstyle\frac13 }
\mdef\qtr{\textstyle\frac14 }
\let\meet\wedge
\let\join\vee
\let\dn\downarrow
\newcommand{\op}{^{\mathrm{op}}}
\newcommand{\co}{^{\mathrm{co}}}
\newcommand{\coop}{^{\mathrm{coop}}}
\let\adj\dashv
\SelectTips{cm}{}
\newdir{ >}{{}*!/-10pt/\dir{>}}    % extra spacing for tail arrows in XYpic
\makeatother
\newcommand{\pushout}[1][dr]{\save*!/#1+1.2pc/#1:(1,-1)@^{|-}\restore}
\newcommand{\pullback}[1][dr]{\save*!/#1-1.2pc/#1:(-1,1)@^{|-}\restore}
\makeatletter
\let\iso\cong
\let\eqv\simeq
\let\cng\equiv
\mdef\Id{\mathrm{Id}}
\mdef\id{\mathrm{id}}
\alwaysmath{ell}
\alwaysmath{infty}
\let\oo\infty
\alwaysmath{odot}
\def\frc#1/#2.{\frac{#1}{#2}}   % \frc x^2+1 / x^2-1 .
\mdef\ten{\mathrel{\otimes}}
\let\bigten\bigotimes
\mdef\sqten{\mathrel{\boxtimes}}
\def\lt{<}                      % For iTex compatibility
\def\gt{>}

%% OPERATORS
\DeclareMathOperator\lan{Lan}
\DeclareMathOperator\ran{Ran}
\DeclareMathOperator\colim{colim}
\DeclareMathOperator\coeq{coeq}
\DeclareMathOperator\eq{eq}
\DeclareMathOperator\Tot{Tot}
\DeclareMathOperator\cosk{cosk}
\DeclareMathOperator\sk{sk}
%\DeclareMathOperator\im{im}
\DeclareMathOperator\Spec{Spec}
\DeclareMathOperator\Ho{Ho}
\DeclareMathOperator\Aut{Aut}
\DeclareMathOperator\End{End}
\DeclareMathOperator\Hom{Hom}
\DeclareMathOperator\Map{Map}

%% ARROWS
% \to already exists
\newcommand{\too}[1][]{\ensuremath{\overset{#1}{\longrightarrow}}}
\newcommand{\ot}{\ensuremath{\leftarrow}}
\newcommand{\oot}[1][]{\ensuremath{\overset{#1}{\longleftarrow}}}
\let\toot\rightleftarrows
\let\otto\leftrightarrows
\let\Impl\Rightarrow
\let\imp\Rightarrow
\let\toto\rightrightarrows
\let\into\hookrightarrow
\let\xinto\xhookrightarrow
\mdef\we{\overset{\sim}{\longrightarrow}}
\mdef\leftwe{\overset{\sim}{\longleftarrow}}
\let\mono\rightarrowtail
\let\leftmono\leftarrowtail
\let\cof\rightarrowtail
\let\leftcof\leftarrowtail
\let\epi\twoheadrightarrow
\let\leftepi\twoheadleftarrow
\let\fib\twoheadrightarrow
\let\leftfib\twoheadleftarrow
\let\cohto\rightsquigarrow
\let\maps\colon
\newcommand{\spam}{\,:\!}       % \maps for left arrows
\def\acof{\mathrel{\mathrlap{\hspace{3pt}\raisebox{4pt}{$\scriptscriptstyle\sim$}}\mathord{\rightarrowtail}}}

% diagxy redefines \to, along with \toleft, \two, \epi, and \mon.

%% EXTENSIBLE ARROWS
\let\xto\xrightarrow
\let\xot\xleftarrow
% See Voss' Mathmode.tex for instructions on how to create new
% extensible arrows.
\def\rightarrowtailfill@{\arrowfill@{\Yright\joinrel\relbar}\relbar\rightarrow}
\newcommand\xrightarrowtail[2][]{\ext@arrow 0055{\rightarrowtailfill@}{#1}{#2}}
\let\xmono\xrightarrowtail
\let\xcof\xrightarrowtail
\def\twoheadrightarrowfill@{\arrowfill@{\relbar\joinrel\relbar}\relbar\twoheadrightarrow}
\newcommand\xtwoheadrightarrow[2][]{\ext@arrow 0055{\twoheadrightarrowfill@}{#1}{#2}}
\let\xepi\xtwoheadrightarrow
\let\xfib\xtwoheadrightarrow
% Let's leave the left-going ones until I need them.

%% EXTENSIBLE SLASHED ARROWS
% Making extensible slashed arrows, by modifying the underlying AMS code.
% Arguments are:
% 1 = arrowhead on the left (\relbar or \Relbar if none)
% 2 = fill character (usually \relbar or \Relbar)
% 3 = slash character (such as \mapstochar or \Mapstochar)
% 4 = arrowhead on the left (\relbar or \Relbar if none)
% 5 = display mode (\displaystyle etc)
\def\slashedarrowfill@#1#2#3#4#5{%
  $\m@th\thickmuskip0mu\medmuskip\thickmuskip\thinmuskip\thickmuskip
   \relax#5#1\mkern-7mu%
   \cleaders\hbox{$#5\mkern-2mu#2\mkern-2mu$}\hfill
   \mathclap{#3}\mathclap{#2}%
   \cleaders\hbox{$#5\mkern-2mu#2\mkern-2mu$}\hfill
   \mkern-7mu#4$%
}
% Here's the idea: \<slashed>arrowfill@ should be a box containing
% some stretchable space that is the "middle of the arrow".  This
% space is created as a "leader" using \cleader<thing>\hfill, which
% fills an \hfill of space with copies of <thing>.  Here instead of
% just one \cleader, we use two, with the slash in between (and an
% extra copy of the filler, to avoid extra space around the slash).
\def\rightslashedarrowfill@{%
  \slashedarrowfill@\relbar\relbar\mapstochar\rightarrow}
\newcommand\xslashedrightarrow[2][]{%
  \ext@arrow 0055{\rightslashedarrowfill@}{#1}{#2}}
\mdef\hto{\xslashedrightarrow{}}
\mdef\htoo{\xslashedrightarrow{\quad}}
\let\xhto\xslashedrightarrow

%% To get a slashed arrow in XYmatrix, do
% \[\xymatrix{A \ar[r]|-@{|} & B}\]
%% To get it in diagxy, do
% \morphism/{@{>}|-*@{|}}/[A`B;p]

%% Here is an \hto for diagxy:
% \def\htopppp/#1/<#2>^#3_#4{\:%
% \ifnum#2=0%
%    \setwdth{#3}{#4}\deltax=\wdth \divide \deltax by \ul%
%    \advance \deltax by \defaultmargin  \ratchet{\deltax}{100}%
% \else \deltax #2%
% \fi%
% \xy\ar@{#1}|-@{|}^{#3}_{#4}(\deltax,0) \endxy%
% \:}%
% \def\htoppp/#1/<#2>^#3{\ifnextchar_{\htopppp/#1/<#2>^{#3}}{\htopppp/#1/<#2>^{#3}_{}}}%
% \def\htopp/#1/<#2>{\ifnextchar^{\htoppp/#1/<#2>}{\htoppp/#1/<#2>^{}}}%
% \def\htoop/#1/{\ifnextchar<{\htopp/#1/}{\htopp/#1/<0>}}%
% \def\hto{\ifnextchar/{\htoop}{\htoop/>/}}%

% LABELED ISOMORPHISMS
\def\xiso#1{\mathrel{\mathrlap{\smash{\xto[\smash{\raisebox{1.3mm}{$\scriptstyle\sim$}}]{#1}}}\hphantom{\xto{#1}}}}
\def\toiso{\xto{\smash{\raisebox{-.5mm}{$\scriptstyle\sim$}}}}

% SHADOWS
\def\shvar#1#2{{\ensuremath{%
  \hspace{1mm}\makebox[-1mm]{$#1\langle$}\makebox[0mm]{$#1\langle$}\hspace{1mm}%
  {#2}%
  \makebox[1mm]{$#1\rangle$}\makebox[0mm]{$#1\rangle$}%
}}}
\def\sh{\shvar{}}
\def\scriptsh{\shvar{\scriptstyle}}
\def\bigsh{\shvar{\big}}
\def\Bigsh{\shvar{\Big}}
\def\biggsh{\shvar{\bigg}}
\def\Biggsh{\shvar{\Bigg}}

% TYPING JUDGMENTS
% Call this macro as \jd{x:A, y:B |- c:C}.  It adds (what I think is)
% appropriate spacing, plus auto-sized parentheses around each typing judgment.
\def\jd#1{\@jd#1\ej}
\def\@jd#1|-#2\ej{\@@jd#1,,\;\vdash\;\left(#2\right)}
\def\@@jd#1,{\@ifmtarg{#1}{\let\next=\relax}{\left(#1\right)\let\next=\@@@jd}\next}
\def\@@@jd#1,{\@ifmtarg{#1}{\let\next=\relax}{,\,\left(#1\right)\let\next=\@@@jd}\next}
% Here's a version which puts a line break before the turnstyle.
\def\jdm#1{\@jdm#1\ej}
\def\@jdm#1|-#2\ej{\@@jd#1,,\\\vdash\;\left(#2\right)}
% Make an actual comma that doesn't separate typing judgments (e.g. A,B,C : Type).
\def\cm{,}

%% SKIPIT in TikZ
% See http://tex.stackexchange.com/questions/3513/draw-only-some-segments-of-a-path-in-tikz
\long\def\my@drawfill#1#2;{%
\@skipfalse
\fill[#1,draw=none] #2;
\@skiptrue
\draw[#1,fill=none] #2;
}
\newif\if@skip
\newcommand{\skipit}[1]{\if@skip\else#1\fi}
\newcommand{\drawfill}[1][]{\my@drawfill{#1}}

%% TODO: This \autoref in TAC doesn't work with figures (and anything
%% else other than theorems).

%%%% THEOREM-TYPE ENVIRONMENTS, hacked to
%%% (a) number all with the same numbers, and
%%% (b) have the right names.
%% The following code should work in TAC or out of it, and with
%% hyperref or without it.  In all cases, we use \label to label
%% things and \autoref to refer to them (ordinary \ref declines to
%% include names).  The non-hyperref label and reference hack is from
%% Mike Mandell, I believe.
\newif\ifhyperref
\@ifpackageloaded{hyperref}{\hyperreftrue}{\hyperreffalse}
\iftac
  %% In the TAC style, all theorems are actually subsections.  So
  %% the hyperref \autoref doesn't work and we have to use our own code
  %% in any case.  We also have to hook into the \state macros instead
  %% of \@thm since those are what know about the current theorem type.
  \let\your@state\state
  \def\state#1{\my@state#1}
  \def\my@state#1.{\gdef\currthmtype{#1}\your@state{#1.}}
  \let\your@staterm\staterm
  \def\staterm#1{\my@staterm#1}
  \def\my@staterm#1.{\gdef\currthmtype{#1}\your@staterm{#1.}}
  \let\@defthm\newtheorem
  \def\switchtotheoremrm{\let\@defthm\newtheoremrm}
  \def\defthm#1#2#3{\@defthm{#1}{#2}} % Ignore the third argument (for cleveref only)
  % Start out \currthmtype as empty
  \def\currthmtype{}
  % In a bit, we're going to redefine \label so that \label{athm} will
  % also make a label "label@name@athm" which is the current value of
  % \currthmtype.  Now \autoref{athm} just adds a reference to this in
  % front of the reference.
  \ifhyperref
    \def\autoref#1{\ref*{label@name@#1}~\ref{#1}}
  \else
    \def\autoref#1{\ref{label@name@#1}~\ref{#1}}
  \fi
  % This has to go AFTER the \begin{document} because apparently
  % hyperref resets the definition of \label at that point.
  \AtBeginDocument{%
    % Save the old definition of \label
    \let\old@label\label%
    % Redefine \label so that \label{athm} will also make a label
    % "label@name@athm" which is the current value of \currthmtype.
    \def\label#1{%
      {\let\your@currentlabel\@currentlabel%
        \edef\@currentlabel{\currthmtype}%
        \old@label{label@name@#1}}%
      \old@label{#1}}
  }
  \let\cref\autoref
\else\ifcref
  % Cleveref does most of it for us.
  \def\defthm#1#2#3{%
    %% Ensure all theorem types are numbered with the same counter
    \newaliascnt{#1}{thm}
    \newtheorem{#1}[#1]{#2}
    \aliascntresetthe{#1}
    %% This command tells cleveref's \cref what to call things
    \crefname{#1}{#2}{#3}% following brace must be on separate line to support poorman cleveref sed file
  }
  % This one has its own number (e.g. exercises)
  \def\defnewthm#1#2#3{%
    \newtheorem{#1}{#2}[section]
    %% This command tells cleveref's \cref what to call things
    \crefname{#1}{#2}{#3}% following brace must be on separate line to support poorman cleveref sed file
  }
  % \let\autoref\cref  % May want to use \autoref for xr-ed links
\else
  % In non-TAC styles without cleveref, theorems have their own counters and so the
  % hyperref \autoref works, if hyperref is loaded.
  \ifhyperref
    %% If we have hyperref, then we have to make sure all the theorem
    %% types appear to use different counters so that hyperref can tell
    %% them apart.  However, we want them actually to use the same
    %% counter, so we don't have both Theorem 9.1 and Definition 9.1.
    \def\defthm#1#2#3{% Ignore the third argument (for cleveref only)
      %% All types of theorems are number inside sections
      \newtheorem{#1}{#2}[section]%
      %% This command tells hyperref's \autoref what to call things
      \expandafter\def\csname #1autorefname\endcsname{#2}%
      %% This makes all the theorem counters actually the same counter
      \expandafter\let\csname c@#1\endcsname\c@thm}
  \else
    %% Without hyperref, we have to roll our own.  This code is due to
    %% Mike Mandell.  First, all theorems can now "officially" use the
    %% same counter.
    \def\defthm#1#2#3{\newtheorem{#1}[thm]{#2}} % Ignore the third argument (for cleveref only)
    %% Save the label- and theorem-making commands
    \ifx\SK@label\undefined\let\SK@label\label\fi
    \let\old@label\label
    \let\your@thm\@thm
    %% Save the current type of theorem whenever we start one
    \def\@thm#1#2#3{\gdef\currthmtype{#3}\your@thm{#1}{#2}{#3}}
    %% Start that out as empty
    \def\currthmtype{}
    %% Redefine \label so that \label{athm} defines, in addition to a
    %% label "athm" pointing to "9.1," a label "athm@" pointing to
    %% "Theorem 9.1."
    \def\label#1{{\let\your@currentlabel\@currentlabel\def\@currentlabel%
        {\currthmtype~\your@currentlabel}%
        \SK@label{#1@}}\old@label{#1}}
    %% Now \autoref just looks at "athm@" instead of "athm."
    \def\autoref#1{\ref{#1@}}
  \fi
  \let\cref\autoref
\fi\fi

%% Now the code that works in all cases.  Note that TAC allows the
%% optional arguments, but ignores them.  It also defines environments
%% called "theorem," etc.
\newtheorem{thm}{Theorem}[section]
\ifcref
  \crefname{thm}{Theorem}{Theorems}
\else
  \newcommand{\thmautorefname}{Theorem}
\fi
\defthm{cor}{Corollary}{Corollaries}
\defthm{prop}{Proposition}{Propositions}
\defthm{lem}{Lemma}{Lemmas}
\defthm{sch}{Scholium}{Scholia}
\defthm{assume}{Assumption}{Assumptions}
\defthm{claim}{Claim}{Claims}
\defthm{conj}{Conjecture}{Conjectures}
\defthm{hyp}{Hypothesis}{Hypotheses}
\defthm{princ}{Principle}{Principles}
\defthm{conv}{Convention}{Conventions}
\defthm{defnthm}{Definition/Theorem}{Definition/Theorems}
\iftac\switchtotheoremrm\else\theoremstyle{definition}\fi
\defthm{defn}{Definition}{Definitions}
\defthm{notn}{Notation}{Notations}
\iftac\switchtotheoremrm\else\theoremstyle{remark}\fi
\defthm{rmk}{Remark}{Remarks}
\defthm{eg}{Example}{Examples}
\defthm{egs}{Examples}{Examples}
\defthm{ceg}{Counterexample}{Counterexamples}
\defnewthm{@ex}{Exercise}{Exercises}

\ifcref
  % Display format for sections
  \crefformat{section}{\S#2#1#3}
  \Crefformat{section}{Section~#2#1#3}
  \crefrangeformat{section}{\S\S#3#1#4--#5#2#6}
  \Crefrangeformat{section}{Sections~#3#1#4--#5#2#6}
  \crefmultiformat{section}{\S\S#2#1#3}{ and~#2#1#3}{, #2#1#3}{ and~#2#1#3}
  \Crefmultiformat{section}{Sections~#2#1#3}{ and~#2#1#3}{, #2#1#3}{ and~#2#1#3}
  \crefrangemultiformat{section}{\S\S#3#1#4--#5#2#6}{ and~#3#1#4--#5#2#6}{, #3#1#4--#5#2#6}{ and~#3#1#4--#5#2#6}
  \Crefrangemultiformat{section}{Sections~#3#1#4--#5#2#6}{ and~#3#1#4--#5#2#6}{, #3#1#4--#5#2#6}{ and~#3#1#4--#5#2#6}
  % Display format for appendices
  \crefformat{appendix}{Appendix~#2#1#3}
  \Crefformat{appendix}{Appendix~#2#1#3}
  \crefrangeformat{appendix}{Appendices~#3#1#4--#5#2#6}
  \Crefrangeformat{appendix}{Appendices~#3#1#4--#5#2#6}
  \crefmultiformat{appendix}{Appendices~#2#1#3}{ and~#2#1#3}{, #2#1#3}{ and~#2#1#3}
  \Crefmultiformat{appendix}{Appendices~#2#1#3}{ and~#2#1#3}{, #2#1#3}{ and~#2#1#3}
  \crefrangemultiformat{appendix}{Appendices~#3#1#4--#5#2#6}{ and~#3#1#4--#5#2#6}{, #3#1#4--#5#2#6}{ and~#3#1#4--#5#2#6}
  \Crefrangemultiformat{appendix}{Appendices~#3#1#4--#5#2#6}{ and~#3#1#4--#5#2#6}{, #3#1#4--#5#2#6}{ and~#3#1#4--#5#2#6}
  \crefformat{subappendix}{\S#2#1#3}
  \Crefformat{subappendix}{Section~#2#1#3}
  \crefrangeformat{subappendix}{\S\S#3#1#4--#5#2#6}
  \Crefrangeformat{subappendix}{Sections~#3#1#4--#5#2#6}
  \crefmultiformat{subappendix}{\S\S#2#1#3}{ and~#2#1#3}{, #2#1#3}{ and~#2#1#3}
  \Crefmultiformat{subappendix}{Sections~#2#1#3}{ and~#2#1#3}{, #2#1#3}{ and~#2#1#3}
  \crefrangemultiformat{subappendix}{\S\S#3#1#4--#5#2#6}{ and~#3#1#4--#5#2#6}{, #3#1#4--#5#2#6}{ and~#3#1#4--#5#2#6}
  \Crefrangemultiformat{subappendix}{Sections~#3#1#4--#5#2#6}{ and~#3#1#4--#5#2#6}{, #3#1#4--#5#2#6}{ and~#3#1#4--#5#2#6}
  % Display format for parts
  \crefname{part}{Part}{Parts}
  % Display format for figures
  \crefname{figure}{Figure}{Figures}
\fi


% \qedhere for TAC
\iftac
  \let\qed\endproof
  \let\your@endproof\endproof
  \def\my@endproof{\your@endproof}
  \def\endproof{\my@endproof\gdef\my@endproof{\your@endproof}}
  \def\qedhere{\tag*{\endproofbox}\gdef\my@endproof{\relax}}
\fi

% Make the optional arguments to TAC's \proof behave like everyone else's
\iftac
  \def\pr@@f[#1]{\subsubsection*{\sc #1.}}
\fi

% How to get QED symbols inside equations at the end of the statements
% of theorems.  AMS LaTeX knows how to do this inside equations at the
% end of *proofs* with \qedhere, and at the end of the statement of a
% theorem with \qed (meaning no proof will be given), but it can't
% seem to combine the two.  Use this instead.
\def\thmqedhere{\expandafter\csname\csname @currenvir\endcsname @qed\endcsname}

\ifbeamer\else
  \renewcommand{\theenumi}{(\alph{enumi})}
  \renewcommand{\labelenumi}{\theenumi}
  \renewcommand{\theenumii}{(\roman{enumii})}
  \renewcommand{\labelenumii}{\theenumii}
\fi

% Left margins for enumitem
\ifbeamer\else
  \setitemize[1]{leftmargin=2em}
  \setenumerate[1]{leftmargin=*}
\fi

% Also number formulas with the theorem counter
\iftac
  \let\c@equation\c@subsection
\else
  \let\c@equation\c@thm
\fi
\numberwithin{equation}{section}

% Only show numbers for equations that are actually referenced (or
% whose tags are specified manually) - uses mathtools.  All equations
% need to be referenced with \eqref, not \ref, for this to work!
\ifcref\else
  \@ifpackageloaded{mathtools}{\mathtoolsset{showonlyrefs,showmanualtags}}{}
\fi

% GREEK LETTERS, ETC.
\alwaysmath{alpha}
\alwaysmath{beta}
\alwaysmath{gamma}
\alwaysmath{Gamma}
\alwaysmath{delta}
\alwaysmath{Delta}
\alwaysmath{epsilon}
\mdef\ep{\varepsilon}
\alwaysmath{zeta}
\alwaysmath{eta}
\alwaysmath{theta}
\alwaysmath{Theta}
\alwaysmath{iota}
\alwaysmath{kappa}
\alwaysmath{lambda}
\alwaysmath{Lambda}
\alwaysmath{mu}
\alwaysmath{nu}
\alwaysmath{xi}
\alwaysmath{pi}
\alwaysmath{rho}
\alwaysmath{sigma}
\alwaysmath{Sigma}
\alwaysmath{tau}
\alwaysmath{upsilon}
\alwaysmath{Upsilon}
\alwaysmath{phi}
\alwaysmath{Pi}
\alwaysmath{Phi}
\mdef\ph{\varphi}
\alwaysmath{chi}
\alwaysmath{psi}
\alwaysmath{Psi}
\alwaysmath{omega}
\alwaysmath{Omega}
\let\al\alpha
\let\be\beta
\let\gm\gamma
\let\Gm\Gamma
\let\de\delta
\let\De\Delta
\let\si\sigma
\let\Si\Sigma
\let\om\omega
\let\ka\kappa
\let\la\lambda
\let\La\Lambda
\let\ze\zeta
\let\th\theta
\let\Th\Theta
\let\vth\vartheta
\let\Om\Omega

%% Include or exclude solutions
% This code is basically copied from version.sty, except that when the
% solutions are included, we put them in a `proof' environment as
% well.  To include solutions, say \includesolutions; to exclude them
% say \excludesolutions.
% \begingroup
% 
% \catcode`{=12\relax\catcode`}=12\relax%
% \catcode`(=1\relax \catcode`)=2\relax%
% \gdef\includesolutions(\newenvironment(soln)(\begin(proof)[Solution])(\end(proof)))%
% \gdef\excludesolutions(%
%   \gdef\soln(\@bsphack\catcode`{=12\relax\catcode`}=12\relax\soln@NOTE)%
%   \long\gdef\soln@NOTE##1\end{soln}(\solnEND@NOTE)%
%   \gdef\solnEND@NOTE(\@esphack\end(soln))%
% )%
% \endgroup

\makeatother

% Local Variables:
% mode: latex
% TeX-master: ""
% End:

\title{Categorical logic from a categorical point of view}
\author{}
\date{\today}
\autodefs{\bSet}
\let\types\vdash
\begin{document}
\maketitle
\tableofcontents

\section{Appetizer: inverses in group objects}
\label{sec:intro}

In this section we consider an extended example.
We do not expect the reader to understand it very deeply, but we hope it will give some motivation for what follows, as well as a taste of the power and flexibility of categorical logic as a tool for category theory.

Our example will consist of several varations on the following theorem:

\begin{thm}
  If a monoid has inverses (hence is a group), then those inverses are unique.
\end{thm}

When ``monoid'' and ``group'' have their usual meaning, namely sets equipped with structure, the proof is easy.
For any $x$, if $i(x)$ and $j(x)$ are both two-sided inverse of $x$, then we have
\[ i(x) = i(x) \cdot e = i(x) \cdot (x \cdot j(x)) = (i(x)\cdot x)\cdot j(x) = e\cdot j(x) = j(x) \]
However, the theorem is true much more generally than this.
We consider first the case of monoid/group objects in a category with products.
A \emph{monoid object} is an object $M$ together with maps $m:M\times M \to M$ and $e:1\to M$ satisfying associativity and unitality axioms:
\begin{equation}
  \vcenter{\xymatrix{
      M\times M\times M\ar[r]^-{1\times m}\ar[d]_{m\times 1} &
      M\times M\ar[d]^m\\
      M\times M\ar[r]_m &
      M
    }}
  \qquad
  \vcenter{\xymatrix{ M \ar[r]^-{(1,e)} \ar[dr]_{1} &
    M\times M \ar[d]_m & M \ar[l]_-{(e,1)} \ar[dl]^{1} \\
    & M }}
\end{equation}
An \emph{inverse operator} for a monoid object is a map $i:M\to M$ such that
\begin{equation}
  \vcenter{\xymatrix@C=1pc{& M\times M \ar[rr]^{i\times 1} && M\times M \ar[dr]^m \\
      M \ar[ur]^{\Delta} \ar[dr]_{\Delta} \ar[rr]^{!} && 1 \ar[rr]^{e} && M \\
      & M\times M \ar[rr]_{1\times i} && M\times M \ar[ur]_{m}}}
\end{equation}
The internalized claim, then, is that any two inverse operators are equal.
The \emph{internal logic of a category with products} allows us to prove this by using essentially the same argument that we did in the case of ordinary monoids in \bSet.
The morphisms $m$ and $e$ are represented in this logic by the notations
\begin{mathpar}
  x:M,y:M \types x\cdot y :M \and
  \types e:M.
\end{mathpar}
Don't worry if this notation doesn't make a whole lot of sense yet.
The symbol $\types$ (called a ``turnstile'') is the logic version of a morphism arrow $\to$, and the entire notation is called a \emph{sequent} or a \emph{judgment}.
The fact that $m$ is a morphism $M\times M \to M$ is indicated by the fact that $M$ appears twice to the left of $\types$ and once to the right; the comma ``$,$'' in between $x:M$ and $y:M$ represents the product $\times$, and the variables $x,y$ are there so that we have a good notation ``$x\cdot y$'' for the morphism $m$.
In particular, the notation $x:M,y:M \types x\cdot y :M$ should be bracketed as
\[ ((x:M),(y:M)) \types ((x\cdot y) :M). \]
Similarly, the associativity, unit, and inverse axioms are indicated by the notations
\begin{mathpar}
  x:M,y:M,z:M \types (x\cdot y)\cdot z = x\cdot (y\cdot z) : M \\
  x:M \types x\cdot e = x : M \and
  x:M \types e\cdot x = x : M \\F
  x:M \types x\cdot i(x) = e : M \and
  x:M \types i(x) \cdot x = e : M
\end{mathpar}
Now the \bSet-based proof can be essentially copied in this notation:
\[ x:M \types i(x) = i(x) \cdot e = i(x) \cdot (x \cdot j(x)) = (i(x)\cdot x)\cdot j(x) = e\cdot j(x) = j(x) : M.\]
The essential point is that the notation \emph{looks set-theoretic}, with ``variables'' representing ``elements'', and yet (as we will see) its formal structure is such that it can be interpreted into \emph{any} category with products.
Therefore, writing the proof in this way yields automatically a proof of the general theorem that any two inverse \emph{operators} for a monoid \emph{object} in a category with products are equal.

To be sure, such a proof could be written in standard categorical style using commutative diagrams as well (\cref{ex:fp-inv-uniq}), and experienced category theorists become quite good at translating proofs in this way.
The point of categorical logic is that this sort of work is \emph{so} straightforward that it's actually completely unnecessary to do at all: we can prove a ``meta-theorem'' that does all the work for us.

Before leaving this appetizer section, we mention some further generalizations of this result.
While type theory allows us to use set-like notation to prove facts about any category with finite products, the allowable notation is fairly limited, essentially restricting us to algebraic calculations with variables.
However, if our category has more structure, then we can ``internalize'' more set-theoretic arguments.

As an example, note that for ordinary monoids in sets the uniqueness of inverses can be expressed ``pointwise'' rather than in terms of inverse-assigning operators.
In other words, for each element $x\in M$, if $x$ has two two-sided inverses $y$ and $z$, then $y=z$.
If we think hard enough, we can express this diagrammatically in terms of the category \bSet is to consider the following two sets:
\begin{align*}
  A &= \setof{(x,y,z)\in M^3 | xy=e, yx=e, xz=e, zx=e}\\
  B &= \setof{(y,z)\in M^2 | y=z}
\end{align*}
In other words, $A$ is the set of elements $x$ equipped with two inverses, and $B$ is the set of pairs of equal elements.
Then the uniqueness of pointwise inverses can be expressed by saying there is a commutative diagram
\[ \xymatrix{ A \ar[d] \ar[r] & B \ar[d] \\ M^3 \ar[r]_{\pi_{23}} & M^2 } \]
where the vertical arrows are inclusions and the lower horizontal arrow projects to the second and third components.

This is a statement that makes sense for a monoid object $M$ in any category with finite \emph{limits}.
The object $B$ can be constructed categorically as the equalizer of the two projections $M\times M \toto M$ (which is in fact isomorphic to $M$ itself), while the object $A$ is a ``joint equalizer'' of four parallel pairs, one of which is
\[ \vcenter{\xymatrix{ & M \times M \ar[dr]^m \\
    M\times M\times M \ar[ur]^{\pi_{12}} \ar[dr]_{!} && M \\
    & 1 \ar[ur]_e }} \]
and the others are similar.
We can then try to \emph{prove}, in this generality, that there is a commutative square as above.
We can do this by manipulating arrows, or by appealing to the Yoneda lemma, but we can also use the \emph{internal logic of a category with finite limits}.
This is a syntax like the internal logic for categories with finite products, but which also allows us to \emph{hypothesize equalities}.
The judgment in question is
\begin{equation}\label{eq:pointwise-unique-inverses}
  x:M, y:M, z:M, x\cdot y = e, y\cdot x=e, x\cdot z = e, z\cdot x = e \types y=z.
\end{equation}
As before, the comma binds the most loosely, so this should be read as
\[ ((x:M), (y:M), (z:M), (x\cdot y = e), (y\cdot x=e), (x\cdot z = e), (z\cdot x = e)) \types (y=z). \]
We can prove this by set-like equational reasoning, essentially just as before.
The ``interpretation machine'' then produces from this a morphism $A\to B$, for the objects $A$ and $B$ constructed above.

Next, note that in the category \bSet, the uniqueness of inverses ensures that if every element $x\in M$ has an inverse, then there is a \emph{function} $i:M\to M$ assigning inverses --- even without using the axiom of choice.
(If we define functions as sets of ordered pairs, as is usual in set-theoretic foundations, we could take $i = \setof{(x,y) | xy=e}$; the pointwise uniqueness ensures that this is indeed a function.)
This fact can be expressed in the internal logic of an \emph{elementary topos}.
We postpone the definition of a topos until later; for now we just remark that its structure allows both sides of the turnstile $\types$ to contain \emph{logical formulas} such as $\exists x, \forall y, \phi(x,y)$ rather than just elements and equalities.
In this language we can state and prove the following:
\[ \forall x:M, \exists y:M, x\cdot y = e \land y\cdot x = e \types
\exists i:M^M, \forall x:M, (x\cdot i(x) = e \land i(x)\cdot x = e)
\]
As before, the proof is essentially exactly like the usual set-theoretic one.
Moreover, the interpretation machine allows us to actually extract an ``inverse operator'' morphism in the topos from this proof.
As before, such a result can also be stated and proved using arrows and commutative diagrams, but as the theorems get more complicated, the translation gets more tedious to do by hand, and the advantage of type-theoretic notation becomes greater.

So much for adding extra structure.
In fact, we can also take structure away!
A monoid object can be defined internal to any \emph{monoidal} category, not just a cartesian monoidal one; now the structure maps are $m:M\otimes M\to M$ and $e:I\to M$, and the commutative diagrams are essentially the same.

To define an inverse operator in this case, however, we need some sort of ``diagonal'' $\Delta:M\to M\otimes M$ and also a ``projection'' or ``augmentation'' $\varepsilon:M\to I$.
The most natural hypothesis is that these maps make $M$ into a \emph{comonoid} object, i.e.\ a monoid in the opposite monoidal category, and that the monoid and comonoid structures preserve each other; this is the notion of a \emph{bimonoid} (or ``bialgebra'').
(\cref{ex:cartmon-bimon-uniq}: in a cartesian monoidal category, every object is a bimonoid in a unique way.)

Now given a bimonoid $M$, we can define an ``inverse operator'' --- which in this context is usually called an \emph{antipode} --- to be a map $i:M\to M$ such that
\begin{equation}
  \vcenter{\xymatrix@C=1pc{& M\otimes M \ar[rr]^{i\otimes 1} && M\otimes M \ar[dr]^m \\
      M \ar[ur]^{\Delta} \ar[dr]_{\Delta} \ar[rr]^{\varepsilon} && I \ar[rr]^{e} && M \\
      & M\otimes M \ar[rr]_{1\otimes i} && M\otimes M \ar[ur]_{m}}}
\end{equation}
commutes, where now $\Delta$ and $\varepsilon$ are the comonoid structure of $M$ rather than the diagonal and projection of a cartesian product.
A bimonoid equipped with an antipode is called a \emph{Hopf monoid} (or ``Hopf algebra'').
The obvious question then is, if a bimonoid has two antipodes, are they equal?

In some cases it is possible to apply the previous results directly.
For instance, the category of \emph{(co)commutative} comonoids in a symmetric monoidal category inherits a monoidal structure that turns out to be \emph{cartesian} (\cref{ex:ccmon-cart}), so a cocommutative bimonoid is actually a monoid in a cartesian monoidal category, and we can apply the first version of our result.
Similarly, the category of commutative monoids is cocartesian, so a commutative bimonoid is a comonoid in a cocartesian monoidal category, so we can apply the dual of the first version of our result.
But what if neither the multiplication nor the comultiplication is commutative?

Internal logic is up to this task.
In a monoidal category we can consider judgments with multiple outputs as well as multiple inputs.
This allows us to describe monoids and comonoids in a roughly ``dual'' way:
\begin{alignat*}{2}
  x:M, y:M &\types x\cdot y:M &\qquad x:M &\types (x_1,x_2):(M,M)\\
  &\types e:M &\qquad x:M &\types \cancel{x}:()\\
  x:M,y:M,z:M &\types (x\cdot y)\cdot z = x\cdot (y\cdot z) :M &\qquad x:M &\types (x_{11},x_{12},x_2)=(x_1,x_{21},x_{22}):(M,M,M)\\
  x:M &\types x\cdot e=x:M &\qquad x:M &\types (x_1,\cancel{x_2}) = x:M\\
  x:M &\types e\cdot x=x:M &\qquad x:M &\types (\cancel{x_1},x_2) = x:M
\end{alignat*}
In this language, the bimonoid axioms are
\begin{align*}
  x:M,y:M &\types (x_1\cdot y_1,x_2\cdot y_2) = ((x\cdot y)_1,(x\cdot y)_2) :(M,M)\\
          &\types (e_1,e_2)=(e,e):(M,M)\\
  x:M,y:M &\types \cancel{x\cdot y} = (\cancel{x},\cancel{y}) : ()\\
  &\types \cancel{e}=():()
\end{align*}
And an antipode is a map $x:M \types i(x):M$ such that
\begin{align*}
  x:M &\types x_1\cdot i(x_2) = (e,\cancel{x}) :M\\
  x:M &\types i(x_1)\cdot x_2 = (e,\cancel{x}) :M
\end{align*}
Now if we have another antipode $j$, we can compute
\begin{align*}
  x:M \types i(x)
  &= i(x)\cdot e\\
  &= (i(x_1)\cdot e,\cancel{x_2})\\
  &= i(x_1)\cdot (x_{21} \cdot j(x_{22}))\\
  &= (i(x_1)\cdot x_{21}) \cdot j(x_{22})\\
  &= (\cancel{x_1},e \cdot j(x_{2}))\\
  &= e\cdot j(x)\\
  &= j(x) \qquad :M
\end{align*}
yielding the same result $i=j$.
So even in a non-cartesian situation, we can use a very similar set-like argument, as long as we keep track of where elements get ``duplicated and discarded''.

This concludes our ``appetizer''; I hope it has given you a taste of what categorical logic looks like, and what it can do for category theory.
In the next section we will rewind back to the beginning and start with very simple cases.

\subsection*{Exercises}

\begin{ex}\label{ex:fp-inv-uniq}
  Prove, using arrows and commutative diagrams, that any two inverse operators for a monoid object in a category with finite products are equal.
\end{ex}

\begin{ex}\label{ex:cartmon-bimon-uniq}
  Prove that in a cartesian monoidal category, every object is a bimonoid in a unique way.
\end{ex}

\begin{ex}\label{ex:ccmon-cart}
  Show that the category of cocommutative comonoids in a symmetric monoidal category inherits a monoidal structure, and that this monoidal structure is cartesian.
\end{ex}

\begin{ex}\label{ex:antipode}
  Prove, using arrows and commutative diagrams, that any two antipodes for a bimonoid (not necessarily commutative or cocommutative) are equal.
\end{ex}

\bibliographystyle{alpha}
\bibliography{all}

\end{document}
