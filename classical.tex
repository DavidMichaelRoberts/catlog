\chapter{Classical type theories}
\label{chap:polycats}


\section{Polycategories and linear logic}
\label{sec:cllin}

[Mention linearly distributive categories and $\ast$-autonomous categories.
But don't belabor them, and perhaps just cite references like~\cite{cs:wkdistrib} for their universal characterizations and initiality theorems.]


\section{Classical logic}
\label{sec:classical}

[Cartesian polycategories.  I expect that the polycomposition coming from their definition as generalized polycategories includes the ``mix rule'', and is sufficient for a direct structural proof of cut admissibility.]


\section{Posetal props and symmetric monoidal posets}
\label{sec:proppos-smpos}

Now let us consider sequents $\Gamma\types\Delta$ where the commas on both sides are intended to represent the \emph{same} tensor product $\tensor$.
For simplicity, we assume this tensor product is symmetric, so that we have an exchange rule on both sides.
This leads us to consider different identity and cut rules (in fact, two identity rules):
\begin{mathpar}
  \inferrule*{ }{()\types()}\and
  \inferrule*{\Gamma\types\Delta}{\Gamma,A\types \Delta,A}\and
  \inferrule*[Right=cut]{\Gamma\types\Xi,\Psi \\ \Psi,\Phi \types \Delta}{\Gamma,\Phi \types \Delta,\Xi}
\end{mathpar}
As always, of course, we intend for the cut rule to be admissible, but writing down at this point what it should be helps us build it into other rules appropriately.

To prove an initiality theorem, we need an appropriate categorical structure.
We define a \textbf{prop} to be a symmetric polygraph \cP together with the following data and axioms:
\begin{enumerate}
\item A morphism $\idfunc_{()}:()\to ()$.
\item For every $A\in\cP$, a morphism $\idfunc_A :(A)\to (A)$.
  % For every $f:\Gamma\to\Delta$ and object $A$, a morphism $(f,\idfunc_A):(\Gamma,A) \to (\Delta,A)$.
  % By induction from this and the previous, we have $\idfunc_{\Gamma} :\Gamma\to\Gamma$ for any $\Gamma$.
\item For every $f:\Gamma\to (\Xi,\Psi)$ and $g:(\Psi,\Phi)\to \Delta$, a composite $g\circ_\Psi f : (\Gamma,\Phi) \to (\Xi,\Delta)$.
% \item Identities are invariant under permutation: $\sigma\idfunc_{\Gamma}\sigma = \idfunc_{\sigma\Gamma}$.
\item Composition is invariant under permutation: $\tau(g\circ_\Psi f)\sigma = (\tau g)\circ_\Psi (f\sigma)$ and $g\sigma \circ_\Psi f = g\circ_{\sigma \Psi} \sigma f$.
\item Composition is unital:
  % $g\circ_\Psi \idfunc_\Psi = g$ and $\idfunc_\Psi\circ_\Psi f = f$.
  $g\circ_A \idfunc_A = g$ and $\idfunc_A\circ_A f = f$.
\item Composition is associative: given $f:\Gamma\to (\Xi,\Theta,\Psi)$ and $g:(\Psi,\Phi)\to (\Delta,\Upsilon)$ and $h:(\Upsilon,\Theta,\Lambda)\to \Omega$, we have
  \[h \circ_{\Upsilon,\Theta} (g\circ_\Psi f) = (h\circ_\Upsilon g) \circ_{\Theta,\Psi} f \qquad \text{(as morphisms} (\Gamma,\Phi,\Lambda) \to (\Xi,\Delta,\Omega)). \]
\item Composition is interchanging: $g\circ_{()}f = f\circ_{()}g$ (modulo a symmetry action).
\end{enumerate}
By composing along empty lists, we get a ``tensor product'' of morphisms: if $f:\Gamma\to\Xi$ and $g:\Phi\to\Delta$, then $g\circ_{()}f : (\Gamma,\Phi) \to (\Xi,\Delta)$.
(This sort of operation is what distinguishes a prop from a ``properad''.)
In particular, we can produce identity morphisms for lists: $\idfunc_{(A,B)} = \idfunc_A \circ_{()} \idfunc_B$ and so on.

Similarly we have $f\circ_{()}g : (\Phi,\Gamma) \to (\Delta,\Xi)$, and the interchange axiom means that these two morphisms are related by the appropriate symmetric group actions.
In the special case when $\Gamma=\Phi=\Xi=\Delta=()$, the Eckmann--Hilton argument implies that the morphisms $()\to ()$ form a commutative monoid; thus our props are the original ones of Adams--MacLane~\cite{maclane:natural-assoc,maclane:cat-alg} rather than the ``graphical'' ones of Batanin--Berger~\cite{bb:htapm}.

It may not be obvious that the above axioms do actually give the original notion of prop.
Eventually we will prove this; but in this section we restrict to the posetal case, in which case the axioms are irrelevant.
By a \textbf{posetal prop} we mean a prop in which there is at most one morphism $\Gamma\to\Delta$ for any $\Gamma,\Delta$.
Given a relational polygraph \cG, we define the \textbf{type theory for posetal props under \cG} to have the two identity rules and a Yoneda-ified generator rule, plus a primitive exchange rule on both sides (in \cref{sec:prop-smc} we will make exchange admissible):
\begin{mathpar}
  \inferrule{ }{()\types()}\and
  \inferrule{\Gamma\types\Delta}{\Gamma,A\types \Delta,A}\and
  \inferrule{\Gamma\types\Xi,\Psi \\ f\in\cG(\Psi,\Phi; \Delta)}{\Gamma,\Phi \types \Delta,\Xi}\\
  \inferrule{\Gamma,A,B,\Delta\types \Psi}{\Gamma,B,A,\Delta\types \Psi}\and
  \inferrule{\Gamma\types \Delta,A,B,\Psi}{\Gamma\types \Delta,B,A,\Psi}
\end{mathpar}

\begin{thm}\label{thm:prop-cutadm}
  For any polygraph \cG, the intended cut rule is admissible in the type theory for posetal props under \cG:
  \[ \inferrule*[Right=cut]{\Gamma\types\Xi,\Psi \\ \Psi,\Phi \types \Delta}{\Gamma,\Phi \types \Delta,\Xi}\]
  [TODO: To deal with primitive exchange, this rule may need to incorporate permutations.]
\end{thm}
\begin{proof}
  As always, we induct on the derivation of $\Psi,\Phi \types \Delta$.
  \begin{enumerate}
  \item If it is the empty identity rule $()\types()$, then $\Gamma\types\Xi,\Psi$ is just $\Gamma\types\Xi$ and is also the desired conclusion.
  \item If it ends with the other identity rule, then there are two cases.
    \begin{enumerate}
    \item If $A$ is in $\Phi$, then we have $\Phi=\Phi',A$ and $\Delta=\Delta',A$ and a derivation of $\Psi,\Phi'\types\Delta'$.
      Applying the inductive hypothesis to this we get $\Gamma,\Phi'\types \Xi,\Delta'$, and then applying the identity rule again gives the desired conclusion.
    \item If $A$ is in $\Psi$, then we have $\Psi=\Psi',A$ and $\Delta=\Delta',A$ and a derivation of $\Psi',\Phi\types\Delta'$.
      Now we can inductively cut this with the given $\Gamma\types \Xi,A,\Psi'$ along $\Psi'$ to get $\Gamma,\Phi\types \Xi,A,\Delta'$ which is the desired conclusion.
    \end{enumerate}
  \item Finally, suppose $\Psi,\Phi \types \Delta$ ends with the rule for an $f$.
    Thus, we have $\Delta=\Delta_1,\Delta_2$ and $\Psi=\Psi_1,\Psi_2$ and $\Phi=\Phi_1,\Phi_2$, where $f\in\cG(\Psi_2,\Phi_2,\Upsilon;\Delta_1)$ and a given derivation of $\Psi_1,\Phi_1\types \Delta_2,\Upsilon$.
    We first inductively cut the latter with our given $\Gamma\types \Xi,\Psi_1,\Psi_2$ along $\Psi_1$ to get $\Gamma,\Phi_1\types \Xi,\Psi_2,\Delta_2,\Upsilon$, then apply the $f$ rule on $\Psi_2,\Upsilon$ to get the desired $\Gamma,\Phi_1,\Phi_2\types \Xi,\Delta_1,\Delta_2$ as desired.\qedhere
  \end{enumerate}
\end{proof}

\begin{thm}\label{thm:posprop-initial}
  For any relational polygraph \cG, the free posetal prop generated by \cG is described by the type theory for posetal props under \cG: its objects are those of \cG, and we have $\Gamma\le\Delta$ just when $\Gamma\types\Delta$ is derivable.
\end{thm}
\begin{proof}
  \cref{thm:prop-cutadm} implies that these definitions give us a posetal prop $\F\bSMPos\cG$.
  Now if \cM is any other posetal prop and $P:\cG\to\cM$ is a map of relational polygraphs, to extend $P$ to $\F\bSMPos\cG$ it suffices to check that it preserves the relation.
  This follows by an easy induction on the rules.
\end{proof}

Now, there are actually \emph{three} ways we could imagine introducing tensor products in a prop.
Fortunately, they are all equivalent.

\begin{lem}\label{thm:prop-tensor}
  For any two objects $A$ and $B$ in a prop \cP, the following are equivalent.
  \begin{enumerate}
  \item A morphism $(A,B) \to A\tensor B$, precomposition with which defines bijections\label{item:prop-tensor-left}
    \[ \cP(\Gamma,A\tensor B;\Delta) \to \cP(\Gamma,A,B;\Delta) \]
  \item A morphism $A\tensor B \to (A,B)$, postcomposition with which defines bijections\label{item:prop-tensor-right}
    \[ \cP(\Gamma;\Delta,A\tensor B) \to \cP(\Gamma;\Delta,A,B) \]
  \item Morphisms $(A,B) \to A\tensor B$ and $A\tensor B \to (A,B)$ whose composites in both directions
    \begin{gather*}
      A\tensor B \too (A,B)  \too A\tensor B\\
      (A,B)\too A\tensor B \too (A,B)
    \end{gather*}
    are identities.\label{item:prop-tensor-iso}
  \end{enumerate}
\end{lem}
\begin{proof}
  If we have~\ref{item:prop-tensor-iso}, then pre- or post-composing with the other morphism yields the bijections in~\ref{item:prop-tensor-left} and~\ref{item:prop-tensor-right}.
  On the other hand, given~\ref{item:prop-tensor-left}, the identity $(A,B)\to (A,B)$ must be the composite $(A,B)\to A\tensor B \to (A,B)$ for some unique morphism $A\tensor B \to (A,B)$, and the other composite must be the identity since its precomposite with $(A,B)\to A\tensor B$ is $(A,B)\to A\tensor B$; so~\ref{item:prop-tensor-iso} holds.
  (This is basically the Yoneda lemma.)
  The case of~\ref{item:prop-tensor-right} is dual.
\end{proof}

The case with units is similar.

\begin{lem}\label{thm:prop-unit}
  In a prop \cP, the following are equivalent.
  \begin{enumerate}
  \item A morphism $()\to \one$, precomposition with which defines bijections
    \[\cP(\Gamma,\one;\Delta) \to \cP(\Gamma;\Delta).\]
  \item A morphism $\one\to()$, postcomposition with which defines bijections
    \[\cP(\Gamma;\Delta,\one) \to \cP(\Gamma;\Delta).\]
  \item Morphisms $()\to \one$ and $\one\to()$ whose composites in both directions are identities.\qed
  \end{enumerate}
\end{lem}

\begin{thm}\label{thm:prop-smc}
  There is an equivalence between (1) symmetric monoidal categories and (2) props with tensors and units in the sense of \cref{thm:prop-tensor,thm:prop-unit}.
\end{thm}
\begin{proof}
  Given a symmetric monoidal category \cC, we define a prop \cP by
  \[ \cP(A_1,\dots,A_n ; B_1,\dots,B_m) = \cC(A_1\tensor \cdots\tensor A_n, B_1\tensor\cdots\tensor B_m)\]
  or, more succinctly, $\cP(\Gamma;\Delta) = \cC(\bigtensor\Gamma, \bigtensor\Delta)$.
  Of course, $\bigtensor() = \one$.
  The identity rules come from $\idfunc_\one$ and $f\tensor \idfunc_A$, while the composite of $f:\Gamma\to (\Xi,\Psi)$ and $g:(\Psi,\Phi)\to \Delta$ is $(\bigtensor\Xi \tensor g) \circ (f\tensor \bigtensor \Phi)$.
  Of course, this prop has tensors and units quite trivially.

  Conversely, given a prop with tensors and units, we have an underlying category with an object $\one$ and an operation $\tensor$.
  The functoriality of $\tensor$ is defined on $f:A\to A'$ and $g:B\to B'$ as the composite
  \[ A\tensor B \too (A,B) \xto{(f,\idfunc)} (A',B) \xto{(\idfunc,g)} (A',B') \too A'\tensor B'. \]
  The axioms of a prop ensure that this is the same as if we put $f$ and $g$ in the other order (or we could write simply $(f,g)$ using the yet-to-be-defined polycomposition).
  Functoriality is similarly easy.
  For associativity, we have
  \[ ((A\tensor B)\tensor C) \too (A\tensor B,C) \too (A,B,C) \too (A,B\tensor C) \to (A\tensor (B\tensor C)) \]
  and for unitality we have
  \[ (A\tensor \unit) \too (A,\unit) \too (A) \]
  and dually.
  It is straightforward to check that these are natural isomorphisms and satisfy the necessary axioms.
  For symmetry, we use the fact that our prop is a \emph{symmetric} polygraph and act by symmetry on the morphisms $(A,B)\to A\tensor B$ and $A\tensor B\to (A,B)$, obtaining ``inverse isomorphisms'' $(B,A) \to A\tensor B$ and $A\tensor B \to (B,A)$.
  From these we can show $A\tensor B \cong B\tensor A$ and check naturality and the axioms.

  It is straightforward to check that these two constructions are inverses.
\end{proof}

\cref{thm:prop-tensor,thm:prop-unit} give several ways to introduce $\tensor$ and $\unit$ into the type theory of posetal props.
For a sequent calculus, the symmetrical approach is probably the most natural:
\begin{mathpar}
  \inferrule{\Gamma,A,B\types \Delta}{\Gamma,A\tensor B\types \Delta}\;\tensorL\and
  \inferrule{\Gamma\types \Delta,A,B}{\Gamma\types \Delta,A\tensor B}\;\tensorR\and
  \inferrule{\Gamma\types\Delta}{\Gamma,\one\types\Delta}\;\one L\and
  \inferrule{\Gamma\types\Delta}{\Gamma\types\Delta,\one}\;\one R\and
\end{mathpar}
Of course, we also include the identity and generator rules (the former only for base types coming from \cG), the exchange rules, and the type judgment:
\begin{mathpar}
  \inferrule{ }{()\types()}\and
  \inferrule{\Gamma\types\Delta\\ A\in\cG}{\Gamma,A\types \Delta,A}\and
  \inferrule{\Gamma\types\Xi,\Psi \\ f\in\cG(\Psi,\Phi; \Delta)}{\Gamma,\Phi \types \Delta,\Xi}\\
  \inferrule{\Gamma,A,B,\Delta\types \Psi}{\Gamma,B,A,\Delta\types \Psi}\and
  \inferrule{\Gamma\types \Delta,A,B,\Psi}{\Gamma\types \Delta,B,A,\Psi}\\
  \inferrule{A\in\cG}{\types A\type}\and
  \inferrule{ }{\types \one\type}\and
  \inferrule{\types A\type \\ \types B\type}{\types A\tensor B\type}
\end{mathpar}
We call this the \textbf{classical sequent calculus for symmetric monoidal posets under \cG}.
(``Classical'' because we have multiple formulas on both sides, ``posets'' because we are not yet distinguishing derivations or introducing terms.)

\begin{thm}\label{thm:seqcalc-smpos-invertible}
  All the rules for $\tensor$ and $\one$ in the classical sequent calculus for symmetric monoidal posets under \cG are invertible.
\end{thm}
\begin{proof}
  Suppose we have a derivation of $\Gamma,A\tensor B\types \Delta$; we want a derivation of $\Gamma,A,B\types \Delta$.
  We induct on the given derivation.
  \begin{enumerate}
  \item The easy case is when it ends with $\tensorL$ whose premise is what we want.
  \item If it ends with $\tensorL$ acting on another type, or any of $\tensorR$, $\one L$, or $\one R$, then we induct on the premise and then apply that rule to the result.
  \item If it ends with the identity rule for $C$, then since $C\in \cG$ it can't be $A\tensor B$, so we can again induct on the premise and apply the identity rule afterwards.
  \item Finally, if it ends with a generator rule for $f$, then since the domain of $f$ is a list of objects of $\cG$, none of them can be $A\tensor B$; so $A\tensor B$ must be in the sequent that the generator rule cuts with, and we can induct on that and then apply the generator rule.
  \end{enumerate}
  The other rules $\tensorR$, $\one L$, and $\one R$ are similar.
\end{proof}

\begin{thm}\label{thm:seqcalc-smpos-identity}
  The following general identity rule is admissible in the classical sequent calculus for symmetric monoidal posets under \cG:
  \begin{mathpar}
    \inferrule{\Gamma\types\Delta\\ \types A\type}{\Gamma,A\types \Delta,A}\and
  \end{mathpar}
  In particular, since $()\types()$, we have $A\types A$ whenever $\types A\type$.
\end{thm}
\begin{proof}
  We induct on the derivation of $\types A\type$.
  \begin{enumerate}
  \item If it is from \cG, we apply the postulated identity rule.
  \item If it is $\one$, we have the following derivation:
    \begin{mathpar}
      \inferrule*{\inferrule*{\inferrule*{ }{()\types()}}{() \types \one}}{\one\types\one}
    \end{mathpar}
  \item If it is $A_1\tensor A_2$, we have by induction a derivation $\sD_1$ of $\Gamma,A_1\types,\Delta,A_1$, and then (again by induction) a derivation $\sD_2$ of $\Gamma,A_1,A_2\types \Delta,A_1,A_2$.
    Now we augment this in a similar way:
    \begin{equation*}
      \inferrule*{
        \inferrule*{
          \inferrule*{\sD_2\\\\\vdots}{\Gamma,A_1,A_2\types \Delta,A_1,A_2}
        }{
          \Gamma,A_1\tensor A_2\types \Delta,A_1,A_2
        }}{
        \Gamma,A_1\tensor A_2\types \Delta,A_1\tensor A_2
      }\qedhere
    \end{equation*}
  \end{enumerate}
\end{proof}

\begin{thm}\label{thm:seqcalc-smpos-cutadm}
  The prop cut rule from \cref{thm:prop-cutadm} is admissible in the classical sequent calculus for symmetric monoidal posets under \cG.
\end{thm}
\begin{proof}
  Suppose given derivations of $\Gamma\types\Xi,\Psi$ and $\Psi,\Phi \types \Delta$; we want to derive $\Gamma,\Phi \types \Delta,\Xi$.
  We induct on the derivation of $\Psi,\Phi \types \Delta$.
  \begin{enumerate}
  \item If it ends with an identity or a generator, we do as in \cref{thm:prop-cutadm}.
  \item If it ends with $\tensorR$ or $\one R$, we can simply apply the inductive hypothesis to cut with the premise of that rule.
  \item If it ends with $\tensorL$, there are two cases.
    If the introduced $A\tensor B$ is in $\Phi$, then we can inductively cut the premise along $\Psi$ again.
  \item On the other hand, if the introduced $A\tensor B$ is in $\Psi$, then our other derivation has the form $\Gamma\types\Xi,\Psi',A\tensor B$.
    Since $\tensorR$ is invertible by \cref{thm:seqcalc-smpos-invertible}, we can also derive $\Gamma\types\Xi,\Psi',A,B$, and then cut with the premise $\Psi',A,B,\Phi \types \Delta$ of our $\tensorL$.
  \item The case of $\one L$ is similar.\qedhere
  \end{enumerate}
\end{proof}

\begin{thm}\label{thm:seqcalc-smpos-initial}
  For any relational polygraph \cG, the free symmetric monoidal poset generated by \cG is described by the classical sequent calculus for symmetric monoidal posets under \cG.
\end{thm}
\begin{proof}
  \cref{thm:seqcalc-smpos-identity,thm:seqcalc-smpos-cutadm} imply in particular that the sequent calculus gives us a posetal prop $\F\bSMPos\cG$.
  Moreover, the rules for $\tensor$ and $\one$ give $\F\bSMPos\cG$ the tensor and unit structure of \cref{thm:prop-tensor,thm:prop-unit}, so that it is a symmetric monoidal poset.
  Now if \cM is any other symmetric monoidal poset and $P:\cG\to\cM$ a map of symmetric relational polygraphs, there is a unique extension of $P$ to the objects of $\F\bSMPos\cG$.
  As usual, by induction on the rules of the sequent calculus and the fact that \cM is a posetal prop, this extension preserves the relations $\Gamma\le\Delta$.
  Finally, it also preserves the tensor and unit structure, so it is a strict functor of symmetric monoidal posets.
\end{proof}


\section{Props and symmetric monoidal categories}
\label{sec:prop-smc}

So much for symmetric monoidal posets.
What we really want, however, is a type theory for symmetric monoidal \emph{categories}, in which we can talk about equality of morphisms, and that can also deal with tensors in the codomain.

In our type theories for categories and multicategories in \cref{sec:category-cutadm,sec:multicat-moncat} (before introducing any operations such as products or tensors), we did not have to impose any equivalence relation on the derivations.
However, in the case of props, the interchange rule makes things more complicated.
For instance, if we have $f\in \cG(A;B)$ and $g\in\cG(C;D)$, then in the type theory of \cref{sec:proppos-smpos} are two distinct derivations of a sequent representing $f\circ_{()} g$:
\begin{mathpar}
  \inferrule*[Right=$g$]{
    \inferrule*{
      \inferrule*[Right=$f$]{
        \inferrule*{\inferrule*{ }{()\types()}}{A\types A}
      }{
        A\types B
      }}{
      A,C\types B,C
    }}{
    A,C\types B,D
  }\and
  \inferrule*[Right=$f$]{
    \inferrule*{
      \inferrule*[Right=$g$]{
        \inferrule*{\inferrule*{ }{()\types()}}{C\types C}
      }{
        C\types D
      }}{
      A,C\types A,D
    }}{
    A,C\types B,D
  }\and
\end{mathpar}
If we write down a term calculus whose terms correspond exactly to derivations in this theory, as we usually do, then the desired equality between these two derivations would look something like
\begin{equation}
  x:A, y:C \types (f,\idfunc)((\idfunc,g)(x,y)) \equiv (\idfunc,g)((f,\idfunc)(x,y)) : (B,D)\label{eq:prop-bad-terms}
\end{equation}
Note that unlike the $\beta$- and $\eta$-conversions, the equality~\eqref{eq:prop-bad-terms} is not directional: it makes no sense to regard one or the other side as ``simpler'' or ``more canonical'' than the other.
We would like to avoid having to assume such equalities in $\equiv$, and furthermore the terms appearing in~\eqref{eq:prop-bad-terms} are rather ugly.
One approach to deal with this would be to break the bijection between terms and deductions, in a way that enables us to represent both of the above two derivations by the same term.
However, a better approach is to design a different theory in which there is only \emph{one} derivation of $f\circ_{()} g$, allowing us to maintain the principle that terms correspond uniquely to derivations.

The non-directionality of~\eqref{eq:prop-bad-terms} also makes it unclear how to design a type theory in which one would be permitted but not the other.
Instead we will forbid \emph{both} of them, replacing the generator rule by a ``multi-generator'' rule allowing only a one-step derivation
\[ \inferrule{x:A,y:C \types (x,y):(A,C)}{x:A,y:C \types (f(x),g(y)):(B,D)}\;f,g \]
The intuition in this term notation is of course that $f(x):B$ and $g(y):D$.
We could write it as ``$f(x):B,g(y):D$'', but we choose to tuple the terms up as in $(f(x),g(y)):(B,D)$ for a couple of reasons.
The first reason is that when doing equational reasoning (such as for the antipode calculation), the equalities must relate entire tuples rather than single terms.
The second reason is that in general, we also need to include some ``terms without a type'' (e.g.\ coming from morphisms with empty codomain $()$, which is a judgmental representation of the unit object), and this looks a little nicer when all the terms are grouped together: we write for instance $(f(x),g(y)\mid h(z))$ to mean that $h(z):()$.

There are also, of course, function symbols with \emph{multiple} outputs.
To deal with this case we write $f_1(x)$, $f_2(x)$, and so on for the terms corresponding to all the types in the codomain.
For example, we write the composite of $f:(A,B) \to (C,D)$ with $g:(E,D)\to (F,G)$ as
\[ x:A, y:B, z:E \types (f_1(x,y),g_1(z,f_2(x,y)),g_2(z,f_2(x,y))):(C,F,G) \]
Note that the variables in a context are not literally treated ``linearly'', since they can occur multiple times in the multiple ``components'' of a map $f$.
Instead the ``usages'' of a variable are controlled by the codomain arity of the morphisms applied to them.
In general, we write $\vec f(\vec M)$ for the list of ``all the values of $f$'' applied to the list of arguments $\vec M$.

This does require one further technical device (that will be almost invisible in practice).
Suppose we have $f:()\to (B,C)$, written in our type theory as $()\types (f_1,f_2):(B,C)$, and we compose/tensor it with itself to get a morphism $() \to (B,B,C,C)$.
We would na\"ively write this as $() \types (f_1,f_1,f_2,f_2)$, but this is ambiguous since we can't tell which $f_1$ matches which $f_2$.
We disambiguate the possibilities by writing $() \types (f_1,f'_1,f_2,f'_2)$ or $() \types (f_1,f'_1,f'_2,f_2)$.
Although this issue seems to only arise for morphisms with empty domain and greater than unary codomain, for consistency we formulate the syntax with a label (like $'$) on \emph{every} term former, and simply omit them informally when there is no risk of ambiguity (which includes the vast majority of cases).
We assume given an infinite alphabet of symbols \fA for this purpose (such as $','',''',\dots$, or $1,2,3,\dots$), and we annotate our judgments with a finite subset $\fB\subseteq \fA$ indicating which labels might have been used already in the terms.

Thus, a first approximation to our generator rule is
\[ \inferrule*{
  \Gamma \types^\fB (\vec M^1,\dots,\vec M^n,\vec N \mid \vec Z)
  : (\vec A^1,\dots, \vec A^n,\vec C)\\
  f^1 \in \cG(\vec A^1,\vec D^1)\\\cdots\\
  f^n \in \cG(\vec A^n,\vec D^n)\\\\
  \fa_i\notin \fB\text{ and pairwise distinct}
}{\Gamma \types^{\fB\cup\{\fa_i\}} (\vec f^{1,\fa_1}(\vec M^1),\dots,\vec f^{n,\fa_n}(\vec M^n),\vec N \mid \vec Z)
  : (\vec D^1,\dots, \vec D^n,\vec C)}
\]
Since we have taken over subscripts to indicate the multiple outputs of a single operation, we are using superscripts to distinguish among the $n$ generators $f^1,\dots,f^n$ being applied at once.
For instance, if $f^1\in \cG((A^1_1,A^1_2),(D^1_1,D^1_2,D^1_3))$ then we have
\[ x^1_1:A^1_1, x^1_2:A^1_2 \types (f^1_1(x^1_1,x^1_2),f^1_2(x^1_1,x^1_2),f^1_3(x^1_1,x^1_2)):(D^1_1,D^1_2,D^1_3) \]

We also have to modify the above rule to allow generators with empty target, which get collected into the $\vec Z$'s.
Thus we have
\[ \inferrule*{
  \Gamma \types^\fB (\vec M^1,\dots,\vec M^n,\vec P^1,\dots,\vec P^m,\vec N \mid \vec Z)
  : (\vec A^1,\dots, \vec A^n,\vec B^1,\dots,\vec B^m,\vec C)\\
  f^1 \in \cG(\vec A^1,\vec D^1)\\\cdots\\
  f^n \in \cG(\vec A^n,\vec D^n)\\\\
  g^1 \in \cG(\vec B^1,())\\\cdots\\
  g^m \in \cG(\vec B^n,())\\\\
  \fa_i,\fb_j\notin \fB\text{ and pairwise distinct}
}{\Gamma \types^{\fB\cup\{\fa_i,\fb_j\}} \left(\vec f^{1,\fa_1}(\vec M^1),\dots,\vec f^{n,\fa_n}(\vec M^n),\vec N
    \;\middle|\; \vec g^{1,\fb_1}(\vec P^1),\dots,\vec g^{m,\fb_m}(\vec P^m), \vec Z\right)
  : (\vec D^1,\dots, \vec D^n,\vec C)}
\]
Eventually we will also have to incorporate shuffles into the rulea as in \cref{sec:symmoncat}, but we postpone that for now.
Let us consider instead how to prevent duplication of derivations.
In addition to our desired term
\begin{align}
  x:A, y:C &\types (f(x),g(y)):(B,D)\label{eq:prop-good-term}
  \intertext{we must also be able to write}
  x:A, y:C &\types (f(x),y):(B,C)\label{eq:prop-goodish-term-1}\\
  \intertext{and}
  x:A, y:C &\types (x,g(y)):(A,D)
\end{align}
so how do we prevent ourselves from being able to apply the generator rule again to the latter two, obtaining two more derivations of the same morphism as~\eqref{eq:prop-good-term}?
The idea is to force ourselves to ``apply all functions as soon as possible'': we cannot apply $g$ to $y$ in~\eqref{eq:prop-goodish-term-1} because we \emph{could have} already applied it to produce~\eqref{eq:prop-good-term}.
On the other hand, we could apply $h:(B,C) \to E$ in~\eqref{eq:prop-goodish-term-1} to get
\[ x:A, y:C \types (h(f(x),y)):E \]
because $h$ uses $f$ as one of its inputs and so could not have been applied at the same time as $f$.

In the general case, in a judgment $\Gamma\types^\fB (\vec{M}\mid\vec{W}):\Delta$ we will assign to each label $\fa\in\fB$ (that is, to each ``use'' of a generator morphism) a natural number called its \textbf{depth}.
The depth of $f^\fa$ is defined recursively to be one greater than the maximum depth of all the labels occurring ``outermost'' in the arguments of $f^\fa$, with variables and 0-ary generators considered to have depth $0$.
Thus for instance in $x:A, y:C \types (h(f(x),y)):E$, the depth of $f$ is $1$ since its argument is a variable $x$, hence of depth $0$; while the depth of $h$ is $2$ since its arguments $f(x)$ and $y$ have depth $1$ and $0$ respectively.
(This definition appears to require inspecting the term, but when we make it formal it will be part of the derivation.)

Now we require, as a condition on the application of the generator rule, that each of the new generators appearing in the conclusion (the $f_i$ and $g_j$) is applied to at least one argument that is of maximum depth in the principal premise.
This ensures that the new generators are precisely those of maximum depth in the conclusion.
It also means that we can never apply this rule to 0-ary generators, so we incorporate those into the $\idfunc$ rule







To be precise, we augment our judgments (\emph{not} their terms) by labeling some of the types in the consequent as \emph{active}, denoted by an underline.
The identity rule makes all types active, while the generator rule makes only the outputs of the generators active.
We then restrict the generator rule to require that at least one of the \emph{inputs} of each generator being applied must be active in the premise; this means that none of them could have been applied any sooner, since at least one of their arguments was just introduced by the previous rule.
Thus, our desired derivation
\begin{mathpar}
  \inferrule*[Right={$f,g$}]{\inferrule*{ }{A,C \types \actv{A},\actv{C}}}{A,C\types \actv{B},\actv{D}}
\end{mathpar}
is allowed, while the undesired ones such as
\begin{mathpar}
  \inferrule*[Right=$g$???]{\inferrule*[Right=$f$]{\inferrule*{ }{A,C \types \actv{A},\actv{C}}}{A,C\types \actv{B},C}}{A,C \types B,D}
\end{mathpar}
is not allowed, since in the attempted application of $g$ the input type $C$ is not active.

Of course, this restriction on its own would prevent generators with nullary domain from ever being applied.
Since these can always be applied at the very beginning, we incorporate them into the identity rule.

Finally, since we want to make the exchange rule admissible, we have to build permutations into the rules as well.
As in \cref{sec:symmoncat}, each rule adds only the part of a permutation that hasn't already been specified by the premises.

\begin{figure}
  \centering
  \begin{mathpar}
    \inferrule{
      f_1 \in \cG((),\vec{B_1})\\\cdots\\
      f_n \in \cG((),\vec{B_n})\\\\
      g_1 \in \cG((),())\\\cdots\\
      g_m \in \cG((),())\\\\
      \sigma : {\vec A},{\vec B}_1,\dots, {\vec B}_n \toiso \Delta\\
      \sigma\text{ preserves the relative order of } B_{11}, B_{21},B_{31},\dots, B_{n1}\\
      \text{All types in }\Delta\text{ are active}\\
      \fa_i,\fb_j\in \fB\text{ and pairwise distinct}
    }{\vec x:\vec A\types^\fB
      \left(\sigma\left(\vec x,{\vec f}_1^{\fa_1},\dots,{\vec f}_n^{\fa_n}\right)
      \,\middle|\, g_1^{\fb_1},\dots,g_m^{\fb_m}\right)
      :\Delta}
    \and
    \inferrule{
      \Gamma \types^\fB (\vec M_1,\dots,\vec M_n,\vec P_1,\dots,\vec P_m,\vec N \mid \vec Z)
      : (\vec A_1,\dots, \vec A_n,\vec B_1,\dots,\vec B_m,\vec C)\\
      \text{At least one type in each } \vec A_i, \vec B_j\text{ is active}\\
      f_1 \in \cG(\vec A_1,\vec D_1)\\\cdots\\
      f_n \in \cG(\vec A_n,\vec D_n)\\\\
      g_1 \in \cG(\vec B_1,())\\\cdots\\
      g_m \in \cG(\vec B_m,())\\\\
      \sigma : \vec D_1,\dots,\vec D_n, \vec C \toiso \Delta\\
      \sigma \text{ preserves the relative order of types in } \vec C\\
      \sigma \text{ preserves the relative order of } D_{11}, D_{21}, D_{31},\dots, D_{n1}\\
      \text{In }\Delta\text{ all types from } \vec D_i \text{ are active, all types from } \vec C\text{ are inactive}\\
      \tau \in \mathrm{Shuf}(m,|\vec Z|)\\
      \fa_i,\fb_j\notin \fB\text{ and pairwise distinct}
    }{\Gamma \types^{\fB\cup\{\fa_i,\fb_j\}} %\fa_1,\dots,\fa_n,\fb_1,\dots,\fb_m\}}
      \left(\sigma\left({\vec f}_1^{\fa_1}(\vec M_1),\dots,{\vec f}_n^{\fa_n}(\vec M_n),\vec N\right)
      \,\middle|\, \tau\left({ g}_1^{\fb_1}(\vec P_1),\dots,{ g}_m^{\fb_m}(\vec P_m),\vec Z\right)\right)
      :\Delta
    }
  \end{mathpar}
  \caption{Type theory for props}
  \label{fig:props}
\end{figure}

The resulting \textbf{type theory for props under \cG} is shown in \cref{fig:props}.
Note that nearly all the subscripts shown are for indexing distinct morphisms or contexts, not for distinguishing between the different outputs of the some morphism as above.
The latter is mostly hidden in the notation $\vec f$, though indicated in a couple of places by double subscripts $D_{11}, D_{21},\dots$

The notation is quite heavy, but the idea is not so bad.
For the identity rule, we suppose given a context of variables belonging to types, and a list of generating morphisms with nullary domain.
Then we collect together all the variables in the context together with all the outputs of those of our generators with non-nullary codomain and permute them arbitrarily, except that we maintain the relative order of the first outputs of each generator (this prevents introduction of an extra permutation in the choice of how to order the list of generators).
Combined with our generators with nullary codomain, this gives a list of terms belonging to the appropriately permuted list of types (from the context and the generator outputs), in which all types are active.

For the generator rule, we suppose given a list of generating morphisms with non-nullary domain --- or more precisely two lists, one of morphisms with non-nullary codomain and one of those with nullary codomain.
We also suppose given a term judgment producing the domains of these morphisms concatenated in order, each containing at least one active type, and followed by some other number of types.
Then we collect all the outputs of our generators applied to these inputs, together with the types that were not input into any of them, and permute them arbitrarily, except that we mantain the relative order of the first outputs and of the un-inputted types (again, this prevents introduction of spurious permutations).
We activate all the outputs of our current generators, and deactivate all other types.
Finally, we similarly shuffle together the generators and the given terms of nullary codomain.

[TODO: Examples, unique derivations of terms, admissibility and functoriality of exchange, admissibility of cut, initiality.  Wait until after writing \cref{sec:symmoncat}.]


\newpage

Similarly, we can construct ``presented'' props and symmetric monoidal categories by including arbitrary generators of $\equiv$.
The uniqueness of antipodes in a bimonoid presented in \cref{sec:intro} is an example of this: \cG has one object $M$ and four morphisms
\begin{alignat*}{2}
m&:(M,M)\to M &&\quad\text{(written infix as $m(x,y) = x\cdot y$)}\\
e&:()\to M\\
\comult&:M\to (M,M) &&\quad\text{(written a little abusively as $\comult_i(x) = x_i$)}\\
\counit &:M\to () &&\quad\text{(written $\counit(x) = \cancel{x}$)}
\end{alignat*}
while the generators of $\equiv$ are the bimonoid axioms.

Here is another example.
If $A$ is an object of a prop, a \textbf{dual} of $A$ is an object $A^*$ with morphisms $\eta:()\to (A,A^*)$ and $\counit:(A^*,A)\to ()$ such that $\counit \circ_{A^*} \eta = \idfunc_A$ and $\counit \circ_{A} \eta = \idfunc_A^*$.
(In a symmetric monoidal category this reduces to the usual notion of dual.)
If $A$ has a dual $A^*$, and $f:A\to A$, the \textbf{trace} of $f$ is the composite
\[ \tr(f) = \counit \circ_{A,A^*} (f \circ_A \eta) \]
or equivalently $(\counit \circ_A f)\circ_{A,A^*} \eta$.
In the type theory for props, the axioms of a dual say
\begin{mathpar}
  x:A \types (\eta_1 \mid \counit(\eta_2,x)) \equiv x :A\and
  y:A^* \types (\eta_2 \mid \counit(y,\eta_1)) \equiv y :A^*
\end{mathpar}
while the trace is $(\,\mid \counit(\eta_2,f(\eta_1)))$.
Recall that $\equiv$ is a congruence for substitution; thus the dual axioms mean that \emph{any term} $M$ (appearing even as a sub-term of some other term) can be replaced by $\eta_1^\fa$ if we simultaneously add $\counit^\fb(\eta_2^\fa,M)$ to the list of null terms (here \fa and \fb are fresh labels).
And dually, if $\counit^\fb(\eta_2^\fa,M)$ appears in the null terms, for any term $M$, then it can be removed by replacing $\eta_1^\fa$ (wherever it appears) with $M$.

Now a classical fact about the trace is that it is \emph{cyclic}: if $A$ and $A'$ both have duals $A^*$ and $(A')^*$, and $f:A\to A'$ and $g:A'\to A$, then $\tr(gf) = \tr(fg)$.
To prove this using traditional commutative-diagram reasoning is quite involved.
It does have a pretty and intuitive proof using string diagrams.
However, its proof in the type theory for props is one line:
\[ \tr(gf) = (\,\mid \counit(\eta_2,g(f(\eta_1))))
= (\,\mid \counit(\eta_2,g(\eta'_1)),\counit'(\eta'_2,f(\eta_1)))
= (\,\mid \counit(\eta'_2,f(g(\eta'_1))))
= \tr(fg).
\]
Here $\eta,\counit$ exhibit the dual of $A$, while $\eta',\counit'$ exhibit the dual of $A'$.
The first and last equality are by definition;
the second applies the first duality equation for $A'$ with $x=f(\eta_1)$; and the third applies the first duality equation for $A$ with $x=g(\eta'_1)$.

One thing to note about these examples is that they use only the type theory of props, not its extension to symmetric monoidal categories.
In general, because the underlying prop of a symmetric monoidal category remembers its morphisms both into and out of tensor products, it seems rarely necessary to invoke the actual tensor product objects.

\begin{rmk}
  A \textbf{compact closed} category is a symmetric monoidal category in which every object has a dual.
  By adding appropriate objects, generators, and relations, we can obtain a type theory describing the free compact closed category on a polygraph, or on a graph, or on a category.
  This does not ``solve the coherence problem'' for compact closed categories, however, since with the additional $\equiv$ relations for duals, the terms no longer have an obvious canonical form.
  In fact, it turns out that an explicit description of the free compact closed category on a category can be given, and involves traces in an essential way; see~\cite{kl:cpt}.
\end{rmk}


% Local Variables:
% TeX-master: "catlog"
% End:
